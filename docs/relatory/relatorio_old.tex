\documentclass[12pt]{article}

\usepackage{sbc-template}

\usepackage{graphicx,url}
\usepackage{lscape}

\usepackage[utf8]{inputenc}
\usepackage[T1]{fontenc}
% UTF-8 encoding is recommended by ShareLaTex

\usepackage{algorithm2e}

\sloppy

\title{Anotações sobre Deep Neuroevolution}

\author{Ricardo Henrique Remes de Lima \inst{1}}

\address{Departamento de Informática\\
	Universidade Federal do Paraná (UFPR)\\
  \email{ricardo.hrlima@gmail.com}
}

\begin{document} 

\maketitle

\section{Aprendizagem de Máquina}


Existem tarefas que são difíceis de serem executadas por seres humanos e simples para computadores, pois podem ser facilmente descritas através de um conjunto de regras bem definidas, como por exemplo, um programa de computador que joga xadrez. Porém, os verdadeiros desafios enfrentados pelos computadores estão relacionadas às tarefas que são facilmente executadas por seres humanos, mas são difíceis de serem descritas formalmente, como por exemplo o reconhecimento da fala ou a identificação de faces em imagens \cite{goodfellow2016deep}.


Aprendizagem de Máquina (\textit{Machine Learning} - ML) é uma sub-área da Inteligência Artificial (IA) que estuda técnicas que dá aos computadores a habilidade de aprender sem serem explicitamente programas \cite{simon2013big}. Com relação ao aprendizado, Tom M. Mitchell \cite{mitchell1997machine} fornece uma definição: "Diz-se que um programa de computador aprende pela experiência \textit{E}, com respeito a algum tipo de tarefa \textit{T} e desempenho \textit{P}, se seu desempenho \textit{P} nas tarefas em \textit{T}, na forma medida por \textit{P}, melhoram com a experiência \textit{E}".


As tarefas de aprendizagem de máquina são classificados em três categorias de acordo com a informação (\textit{feedback}) obtido do sistema. São elas: aprendizado supervisionado, aprendizado não supervisionado e aprendizado por reforço.


\subsection{Aprendizado supervisionado}


Na aprendizagem supervisionada os dados utilizados possuem exemplos de entradas e saídas desejadas, e a partir destes exemplos as técnicas aplicadas tem como objetivo aprender a regra geral que mapeia as entradas para as saídas.


\subsection{Aprendizado não supervisionado}


Diferente da aprendizagem supervisionada, os dados utilizados não possuem uma etiqueta, sendo necessário que os algoritmos utilizados encontrem ou criem uma estrutura que relaciona dados semelhantes com o objetivo de identificar padrões.


\subsection{Aprendizado por reforço}


Utilizado em ambientes dinâmicos, onde o objetivo é desempenhar uma determinada tarefa. Enquanto desempenha a tarefa, o algoritmo utilizado recebe recompensas e punições, e utilizando essas informações vai ajustando seu comportamento.


\section{Aprendizagem Profunda}


O desempenho de algoritmos de ML dependem fortemente da representação dos dados que são fornecidos. Por exemplo, um sistema que auxilie na identificação de uma determinada doença, não avalia o paciente diretamente. As informações mais relevantes são fornecidas por um especialista para o sistema, como a presença ou não de febre no paciente. Tais informações são chamadas de \textbf{características} (features). Sem o auxílio do especialista, não seria possível realizar predições úteis utilizando os dados brutos de um exame \cite{goodfellow2016deep}.


Um novo problema surge quando também se torna necessário descobrir quais características devem ser extraídas para um determinado problema. Uma solução para este problema é utilizar AM não apenas o mapeamento entre a representação e a saída, mas também a representação em si. Essa abordagem é chamada de Aprendizado por Representação (\textit{Representation Learning} - RL), e seu objetivo é encontrar um conjunto de características que melhor descreva a relação entre os dados e a saída. 


A Aprendizagem profunda (\textit{Deep Learning} - DL) surgiu para solucionar o principal problema enfrentado em RL, cuja complexidade cresce quando considerados problemas do muno real, onde há grande variação na influência que os dados observados tem. RL introduz representações que são expressas por meio de outras representações mais simples \cite{goodfellow2016deep}.


\subsection{Redes Neurais Profundas}

\section{Neuro-evolução}

\bibliographystyle{plain}
\bibliography{referencias}

\end{document}